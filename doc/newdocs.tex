% vim: fo=aw2tq tw=100

\documentclass[10pt,a4paper,english]{article}
\usepackage{html}

\newcommand{\href}[2]{\htmladdnormallink{#2}{#1}}
\newcommand{\bslash}{\symbol{92}}

\title{DOSE User Interface Documentation}
\author{Stockholm Environmental Institute, Alan Briolat}
\date{}

\begin{document}

\maketitle

\section{Development}

\subsection{Building the F Model}

\subsubsection{Preparing the environment}

\begin{enumerate}

\item Download and run \href{http://www.cygwin.com/setup.exe}{setup.exe} from the 
\href{http://www.cygwin.com/}{Cygwin website}

\item Select the "Archive -> unzip" and "Devel -> make" packages.  \textbf{Note:} Do not install 
Python or MinGW from this screen

\item Click "Next" and let it install the packages - this may take a while!

\item Run the Cygwin shell once to make sure your "home" directory gets created

\item Download 
\href{ftp://ftp.swcp.com/pub/walt/F/FortranTools\_windows\_F.zip}{FortranTools\_windows\_F.zip} from 
the FortranTools FTP server into your Cygwin home directory (e.g.  
\texttt{C:{\bslash}cygwin{\bslash}home{\bslash}your\_username})

\item Open the Cygwin shell

\item Run the following to unpack and install the F compiler: \begin{verbatim}
unzip FortranTools_windows_F.zip
cd FortranTools
sed "s/\r//g" install_fortrantools > install_fortrantools.fixed
./install_fortrantools.fixed \end{verbatim}

\end{enumerate}


\subsubsection{Building the model}

\begin{enumerate}

\item Unzip the source code somewhere

\item Take a look at \texttt{F{\bslash}dose.f90} to see the filename that should be used for the 
input file (and what filename to expect for the output file)

\item Open a Cygwin shell and change directory to where the source code is

\item Run the \texttt{make} command - the compiled program will be called \texttt{dose}

\end{enumerate}


\subsection{Building the User Interface}

foo bar

\end{document}
