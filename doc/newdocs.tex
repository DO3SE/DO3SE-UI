% vim: fo=aw2tq tw=100

\documentclass[10pt,a4paper]{article}
%\usepackage{pslatex}
\usepackage{html}
\hypersetup{colorlinks=true}
\usepackage[vmargin=1in,hcentering]{geometry}

% Hyperlinks suitable for both HTML and PDF
\newcommand{\fhref}[2]{\htmladdnormallinkfoot{#2}{#1}}
% Easier way of using backslash (damn Windows file paths)
\newcommand{\bslash}{\symbol{92}}
% The DO3SE thingy
\newcommand{\dose}{{DO$_3$SE}}

\title{\dose\ Model Documentation}
\author{Stockholm Environmental Institute, Alan Briolat}
\date{}

\begin{document}

\maketitle
%\pagebreak
%\tableofcontents
%\pagebreak

\section{Development}

%\subsection{Requirements}

%\begin{itemize}
%\item Python 2.5
%\item Numpy 1.1.0
%\item wxWidgets 2.8
%\item wxPython 2.8
%\end{itemize}

\subsection{Building the F Model}

\subsubsection*{Preparing the environment}

\begin{enumerate}

\item Install MinGW (Minimal GNU for Windows)
    \begin{enumerate}
    \item Download and run ``MinGW-5.1.4.exe'' from the 
\fhref{http://sourceforge.net/project/showfiles.php?group\_id=2435\&package\_id=240780}{MinGW 
download page}.
    \item Click ``Next'' as necessary (and ``I Agree'' at the license page) --- the default options 
are all that's needed.
    \item Click ``Install'' on the final screen.
    \item Wait for all the necessary files to be downloaded and installed --- this will probably 
take a few minutes on a fast Internet connection.
    \item Click ``Next'' and then ``Finish''.
    \end{enumerate}

\item Install MSYS (A minimal build environment)
    \begin{enumerate}
    \item Download and run ``MSYS-1.0.10.exe'' from the 
    \fhref{http://sourceforge.net/project/showfiles.php?group\_id=2435\&package\_id=24963\&release\_id=89960}{MSYS 
download page}.
    \item Keep clicking ``Next'' and ``Yes'' as appropriate and then click ``Install'' at the final 
screen --- the default options are probably all that's needed.
    \item After the installation, a console window will open for the ``post-install'' process.  
Answer ``y'' when asked if you would like to continue.
    \item The next question asks if MinGW is installed --- answer ``y'' here too.
    \item When asked for the location of the MinGW installation, if you didn't change anything 
during the MinGW install you will want to answer \texttt{c:/mingw} (if you changed the installation 
directory, enter what you changed it to here instead, but remember that the directory separator 
should be \texttt{/}, not \texttt{\bslash}).
    \item ``Press any key to continue . . .''
    \item You probably don't want to read all the technical stuff that comes with MSYS right now, so 
uncheck the two boxes on the final installer screen before clicking ``Finish''.
    \end{enumerate}

\item Install G95 (A free Fortran95 compiler)
    \begin{enumerate}
    \item Go to the \fhref{http://ftp.g95.org/}{G95 downloads page}, click on the first ``Stable 
Version'' link, and then from that section download and run ``Self-extracting Windows x86'' 
(clicking either ``HTTP'' or ``FTP'' next to it).
    \item Leave the ``Destination Folder'' at it's default unless you changed the directory you 
installed MinGW to (in which case you should enter that path here) and click ``Install''.
    \item Click ``Yes'' when asked about installing into the MinGW directory structure.
    \item Also click ``Yes'' when asked about setting PATH and LIBRARY\_PATH.
    \item When the installation is finished, click ``Close''.
    \end{enumerate}

\item Setting the environment variables
    \begin{enumerate}
    \item Open the ``System Properties'' dialog (for example by pressing the ``Win+Break'' key 
combination).
    \item Click on the ``Advanced'' tab, and then on ``Environment Variables''.
    \item Check that the variable ``LIBRARY\_PATH'' exists in either the user or system variables, 
and that it contains \texttt{c:\bslash{}mingw\bslash{}lib} (if your MinGW directory is 
\texttt{c:\bslash{}mingw}, otherwise modify as appropriate).
    \item If it does not, but the variable exists, click on it and then on the ``Edit'' button and 
    add the path, preceeded by a semicolon (``;'').
    \item If the variable does not exist, click ``New'' under system variables, giving the new 
variable the name ``LIBRARY\_PATH'' and the path as the value.
    \item Check that the variable ``PATH'' (or ``Path'') variable under either user or system 
variables contains \texttt{c:\bslash{}mingw\bslash{}bin}.  If it does not, add it in the same way as 
in (d).
    \item Add the MSYS binary path \texttt{c:\bslash{}msys\bslash{}1.0\bslash{}bin} to the ``PATH'' 
variable in the same way as before.
    \end{enumerate}

\end{enumerate}

\subsubsection*{Building the model}

\begin{enumerate}

\item Unzip the source code somewhere.

\item Take a look at \texttt{F{\bslash}dose.f90} to see the filename that should be used for the 
input file (and what filename to expect for the output file).

\item Open a Windows terminal prompt (for example by going to Start---Run... and typing 
\texttt{cmd.exe}).

\item Change directory to where you unzipped the source code (the \texttt{dir} command should show 
several \texttt{.f90} files).

\item Run the \texttt{make} command - the compiled program will be called \texttt{dose.exe}.

\end{enumerate}


%\subsection{Building the User Interface}

%foo bar

\end{document}
