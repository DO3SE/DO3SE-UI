
The F model, when built, is a stand-alone command-line program.  To run it, make sure there is an
input file of the correct filename in the current directory and then execute the program from the
command line:
\begin{lstlisting}
C:\...\DO3SE-src-F-20090620>dir
...
20/06/2009  18:53   <DIR>       .
20/06/2009  18:53   <DIR>       ..
20/06/2009  18:53   <DIR>       F
20/06/2009  18:53               Makefile
20/06/2009  18:56               dose.exe
20/06/2009  18:56               input_newstyle.csv
...
C:\...\DO3SE-src-F-20090620>dose.exe
C:\...\DO3SE-src-F-20090620>dir
...
20/06/2009  18:53   <DIR>       .
20/06/2009  18:53   <DIR>       ..
20/06/2009  18:53   <DIR>       F
20/06/2009  18:53               Makefile
20/06/2009  18:56               dose.exe
20/06/2009  18:56               input_newstyle.csv
20/06/2009  18:59               output.csv
...
\end{lstlisting}

Alternatively, double-click on \verb|dose.exe| to run it---be aware that this will hide any errors
that might have been visible on the command line.

If the output file (\verb|output.csv| by default) is empty, it is most likely that no output columns
have been specified (see the next section).

\subsection{Configuring Input and Output}

The input and output of the F model is configured in \verb|F/dose.f90|.

\begin{itemize}

\item The input and output filenames are set near the bottom of the file, in the \verb|Run_DOSE|
section.

\item The columns to include in the output are set in the \verb|WriteData| subroutine.  Uncomment a
line to include that variable in the output.  The variables and their meanings are documented in
\verb|F/variables.f90|.

\item The input format is defined in the \verb|ReadData| subroutine.  \emph{This should be used for
guidance, but not modified.}  The variables and their meanings are documented in
\verb|F/inputs.f90|.

\item Where certain calculations have multiple implementations, these can be chosen between in the
\verb|Calculate| subroutine.

\end{itemize}

To change any of these options, edit \verb|F/dose.f90| as necessary and save it.  To use the
changes, rebuild the model (see \ref{dev:build}) and run it again.  \emph{If the output filename has
not been changed and the previous results have not been moved, they will be overwritten.}
