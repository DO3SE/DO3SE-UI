\documentclass[10pt,a4paper,english]{article}
\usepackage{babel}
\usepackage{ae}
\usepackage{aeguill}
\usepackage{shortvrb}
\usepackage[latin1]{inputenc}
\usepackage{tabularx}
\usepackage{longtable}
\setlength{\extrarowheight}{2pt}
\usepackage{amsmath}
\usepackage{graphicx}
\usepackage{color}
\usepackage{multirow}
\usepackage{ifthen}
\usepackage[colorlinks=true,linkcolor=blue,urlcolor=blue]{hyperref}
\usepackage[DIV12]{typearea}
%% generator Docutils: http://docutils.sourceforge.net/
\newlength{\admonitionwidth}
\setlength{\admonitionwidth}{0.9\textwidth}
\newlength{\docinfowidth}
\setlength{\docinfowidth}{0.9\textwidth}
\newlength{\locallinewidth}
\newcommand{\optionlistlabel}[1]{\bf #1 \hfill}
\newenvironment{optionlist}[1]
{\begin{list}{}
  {\setlength{\labelwidth}{#1}
   \setlength{\rightmargin}{1cm}
   \setlength{\leftmargin}{\rightmargin}
   \addtolength{\leftmargin}{\labelwidth}
   \addtolength{\leftmargin}{\labelsep}
   \renewcommand{\makelabel}{\optionlistlabel}}
}{\end{list}}
\newlength{\lineblockindentation}
\setlength{\lineblockindentation}{2.5em}
\newenvironment{lineblock}[1]
{\begin{list}{}
  {\setlength{\partopsep}{\parskip}
   \addtolength{\partopsep}{\baselineskip}
   \topsep0pt\itemsep0.15\baselineskip\parsep0pt
   \leftmargin#1}
 \raggedright}
{\end{list}}
% begin: floats for footnotes tweaking.
\setlength{\floatsep}{0.5em}
\setlength{\textfloatsep}{\fill}
\addtolength{\textfloatsep}{3em}
\renewcommand{\textfraction}{0.5}
\renewcommand{\topfraction}{0.5}
\renewcommand{\bottomfraction}{0.5}
\setcounter{totalnumber}{50}
\setcounter{topnumber}{50}
\setcounter{bottomnumber}{50}
% end floats for footnotes
% some commands, that could be overwritten in the style file.
\newcommand{\rubric}[1]{\subsection*{~\hfill {\it #1} \hfill ~}}
\newcommand{\titlereference}[1]{\textsl{#1}}
% end of "some commands"
\title{Building the DOSE model - Windows}
\author{}
\date{}
\hypersetup{
pdftitle={Building the DOSE model - Windows}
}
\raggedbottom
\begin{document}
\maketitle


\setlength{\locallinewidth}{\linewidth}
% vim: fo=aw2tq tw=100 
\hypertarget{contents}{}
\pdfbookmark[0]{Contents}{contents}
\subsubsection*{~\hfill Contents\hfill ~}
\begin{list}{}{}
\item {} \href{\#just-the-f-model}{Just the F model}
\begin{list}{}{}
\item {} \href{\#preparing-the-environment}{Preparing the environment}

\item {} \href{\#building-the-model}{Building the model}

\end{list}

\item {} \href{\#the-user-interface}{The User Interface}
\begin{list}{}{}
\item {} \href{\#id1}{Preparing the environment}

\item {} \href{\#building-the-application}{Building the application}

\end{list}

\end{list}



%___________________________________________________________________________

\hypertarget{just-the-f-model}{}
\pdfbookmark[0]{Just the F model}{just-the-f-model}
\section*{Just the F model}


%___________________________________________________________________________

\hypertarget{preparing-the-environment}{}
\pdfbookmark[1]{Preparing the environment}{preparing-the-environment}
\subsection*{Preparing the environment}
\newcounter{listcnt0}
\begin{list}{\arabic{listcnt0}.}
{
\usecounter{listcnt0}
\setlength{\rightmargin}{\leftmargin}
}
\item {} 
Download and run \href{http://www.cygwin.com/setup.exe}{setup.exe} from the \href{http://www.cygwin.com/}{Cygwin} website.

\item {} 
Select the ``Archive -{\textgreater} unzip'' and ``Devel -{\textgreater} make'' packages.  \textbf{Note:} Do not install Python or
MinGW from this screen

\item {} 
Click ``Next'' and let it install the packages - this may take a while!

\item {} 
Run the Cygwin shell once to make sure your ``home'' directory gets created

\item {} 
Download \href{ftp://ftp.swcp.com/pub/walt/F/FortranTools_windows_F.zip}{FortranTools{\_}windows{\_}F.zip} from the FortranTools FTP server into your Cygwin home
directory (e.g. \texttt{C:{\textbackslash}cygwin{\textbackslash}home{\textbackslash}your{\_}username})

\item {} 
Open the Cygwin shell

\item {} 
Run the following to unpack and install the F compiler:
\begin{quote}{\ttfamily \raggedright \noindent
unzip~FortranTools{\_}windows{\_}F.zip~\\
cd~FortranTools~\\
sed~"s/{\textbackslash}r//g"~install{\_}fortrantools~>~install{\_}fortrantools.fixed~\\
./install{\_}fortrantools.fixed
}\end{quote}

\end{list}


%___________________________________________________________________________

\hypertarget{building-the-model}{}
\pdfbookmark[1]{Building the model}{building-the-model}
\subsection*{Building the model}
\setcounter{listcnt0}{0}
\begin{list}{\arabic{listcnt0}.}
{
\usecounter{listcnt0}
\setlength{\rightmargin}{\leftmargin}
}
\item {} 
Unzip the source code somewhere

\item {} 
Take a look at \texttt{F{\textbackslash}dose.f90} to see the filename that should be used for the input file (and
what filename to expect for the output file)

\item {} 
Open a Cygwin shell and change directory to where the source code is

\item {} 
Run the \texttt{make} command - the compiled program will be called \texttt{dose}

\end{list}


%___________________________________________________________________________

\hypertarget{the-user-interface}{}
\pdfbookmark[0]{The User Interface}{the-user-interface}
\section*{The User Interface}


%___________________________________________________________________________

\hypertarget{id1}{}
\pdfbookmark[1]{Preparing the environment}{id1}
\subsection*{Preparing the environment}
\setcounter{listcnt0}{0}
\begin{list}{\arabic{listcnt0}.}
{
\usecounter{listcnt0}
\setlength{\rightmargin}{\leftmargin}
}
\item {} 
From the \href{http://www.python.org/download/}{Python download} site, download the latest Python Windows Installer and run it,
paying attention to the path it installs into (for example \texttt{C:{\textbackslash}Python25})

\item {} 
Add Python directory to your environment:
\newcounter{listcnt1}
\begin{list}{\arabic{listcnt1}.}
{
\usecounter{listcnt1}
\setlength{\rightmargin}{\leftmargin}
}
\item {} 
Open the Windows Control Panel

\item {} 
Open ``System Properties''

\item {} 
Click on the ``Advanced'' tab

\item {} 
Click ``Environment Variables''

\item {} 
Click on the ``Path'' system variable

\item {} 
Click ``Edit''

\item {} 
Add a semicolon followed by the directory Python was installed in (e.g. \texttt{;c:{\textbackslash}Python25})

\end{list}

\item {} 
From the \href{http://sourceforge.net/project/showfiles.php?group_id=1369&package_id=175103}{Numerical Python download} site, download the latest version of the NumPy Windows
installer (ending in \texttt{.win32-py2.5.msi}) and run it - it should automatically choose the
correct installation directory
\begin{itemize}
\item {} 
\textbf{Note:} The 1.0.4 version is broken (\href{http://cens.ioc.ee/pipermail/f2py-users/2007-November/001487.html}{bug on f2py mailing list})

\end{itemize}

\item {} 
From the \href{http://wxpython.org/download.php\#binaries}{wxPython download} site, download and install the ``win32-unicode'' version of wxPython
for Python 2.5

\item {} 
Go to the \href{http://sourceforge.net/project/showfiles.php?group_id=2435&package_id=240780}{MinGW download} site and download the latest version of the Automated MinGW
Installer

\item {} 
Run the MinGW installer, selecting only a ``Minimal'' install and using the \texttt{C:{\textbackslash}MinGW} directory

\item {} 
From the \href{http://ftp.g95.org/}{G95 download} site, download the ``Self-extracting Windows x86'' version of the G95
Fortran compiler

\item {} 
Run the G95 installer, installing to \texttt{C:{\textbackslash}MinGW} - answer ``Yes'' when prompted about setting the
\texttt{PATH} and \texttt{LIBRARY{\_}PATH} variables

\item {} 
From the \href{http://sourceforge.net/project/showfiles.php?group_id=15583}{py2exe download} site, download the latest version of the py2exe installer (ending in
\texttt{.win32-py2.5.exe}) and run it

\end{list}


%___________________________________________________________________________

\hypertarget{building-the-application}{}
\pdfbookmark[1]{Building the application}{building-the-application}
\subsection*{Building the application}
\setcounter{listcnt0}{0}
\begin{list}{\arabic{listcnt0}.}
{
\usecounter{listcnt0}
\setlength{\rightmargin}{\leftmargin}
}
\item {} 
Extract the DOSE source code somewhere

\item {} 
Open a Windows command prompt (Start menu -{\textgreater} Run..., type ``cmd'', click OK)

\item {} 
Change to the directory the program is in (using ``\texttt{cd}'')

\item {} 
Run the following:
\begin{quote}{\ttfamily \raggedright \noindent
python~setup.py~py2exe~\\
python~fix-dlls.py
}\end{quote}

\item {} 
Go to the ``\texttt{dist}'' directory under where you unpacked the DOSE source code (either in Windows
Explorer or the command prompt) and run \texttt{dose-ui.exe}

\end{list}

\end{document}
