% vim: fo=aw2tq tw=100
\chapter{Developer Guide}


%\section{Software Structure}

%\subsection{F Model}

%\subsection{Graphical User Interface}

%\subsection{Fortran-Python Integration}

%\subsection{Build System}



\section{Development Environment}
\label{dev:env}

To build any of the software, at the very least the software the build system is based on and the 
dependencies for the software must be present.

The build system is a collection of ``Makefiles'', as used by GNU Make.  Therefore any platform on 
which the software is to built must have GNU Make installed.

To build the F model, a Fortran compiler which understands the F standard is needed.  Currently the 
only freely-available such compiler is G95, which is built on GCC (the GNU Compiler Collection), 
therefore both GCC and G95 must be installed in some form to build the F model.

The GUI (Graphical User Interface) is written in the Python programming language, using the 
wxWidgets cross-platform GUI toolkit.  The interface between Python and the F model is created by 
F2PY (Fortran-to-Python) which is part of the NumPy (Numerical Python) library.  The full 
dependencies are:

\begin{itemize}
\item A working environment for building the F model
\item Python $>=$ 2.5 (but not Python 3.x)
\item NumPy $>=$ 1.1.0
\item wxWidgets 2.8
\item wxPython 2.8 (the Python bindings for wxWidgets)
\end{itemize}

The following sections describe how to install the necessary dependencies for both software builds 
on common platforms.


\subsection{Preparing a Windows Environment}
\label{dev:env:windows}

\subsubsection{F Model Dependencies}

G95 for Windows depends on MinGW (``Minimalist GNU for Windows'') to provide GCC.  MinGW also 
provides GNU Make, a port of GCC to the Windows platform, so this must be installed first.  GNU Make 
is provided by MSYS\footnote{MinGW also provides Make, but does not have a full shell environment 
and therefore lacks important utilities.} (``Minimal System''), a basic Linux-like environment for 
Windows. 

\paragraph{Installing MinGW}

\begin{enumerate}

\item Download and run the latest Automated MinGW Installer from the 
\fhref{http://sourceforge.net/project/showfiles.php?group\_id=2435\&package\_id=240780}{MinGW 
download page} (e.g. \verb|MinGW-5.1.4.exe|).

\item Click through the installer; the options are self-explanatory in most cases.

\item When given a choice of which components to install, the ``Minimal'' install is sufficient.  
Select additional features if desired.

\item Click ``Install'' and wait for the necessary files to be downloaded and installed.  This will 
probably take a few minutes on a fast Internet connection.

\end{enumerate}

\paragraph{Installing G95}

\begin{enumerate}

\item Navigate to the \fhref{http://www.g95.org/downloads.shtml}{G95 downloads page}, click on the 
most recent ``Stable Version'' link.

\item Download the installer by clicking one of the links next to ``Self-extracting Windows x86'' 
and run it.

\item Make sure the ``Destination Folder'' is the same as where MinGW was installed to, and click 
``Install'', and click ``Yes'' when notified that it will install into the MinGW directory 
structure.

\item When asked about setting \verb|PATH| and \verb|LIBRARY_PATH| variables, it is not essential to 
answer ``OK'' here, but may be desirable---adding the ``bin'' directory to the \verb|PATH| variable 
removes the need to use the full path to the compilers when using them for other projects.

\end{enumerate}

\paragraph{Installing MSYS}

\begin{enumerate}

\item Download and run the MSYS Base System installer marked as ``Current Release'' from the 
\fhref{http://sourceforge.net/project/showfiles.php?group\_id=2435\&package\_id=24963}{MSYS download 
page} (e.g. \verb|MSYS-1.0.10.exe|).

\item Click through the installer; the options are self-explanatory in most cases.

\item In the command prompt window that appears after installing MSYS, it is safe to answer ``n'' to 
skip the post-install process.

\item It's probably undesirable to read all the technical documents that come with MSYS at this 
point, so uncheck the two boxes on the final screen before clicking ``Finish''.

\end{enumerate}

\paragraph{Adding MSYS to the PATH Variable (Optional)}

It may be desirable to add MSYS to the \verb|PATH| environment variable so it is not necessary to 
use the full path to \verb|make| when running the build system.

\begin{enumerate}

\item Open the ``System Properties'' dialog, for example by pressing the ``Win+Break'' key 
combination or right-clicking on ``My Computer'' and clicking ``Properties''.

\item Click the ``Advanced'' tab, and then the ``Environment Variables'' button.

\item If the \verb|PATH| variable already exists in the ``User variables'' section, click it and 
click ``Edit'', otherwise click ``New'' and use \verb|PATH| as the variable name.

\item Add the path to the MSYS binary path to the value (prefixing it with a semicolon---``;''---if 
a value already exists).  The MSYS binary path should look something like \verb|c:\msys\1.0\bin|.

\end{enumerate}


%\subsubsection{GUI Dependencies}


%\subsection{Preparing a Linux Environment}

%\subsubsection{F Model Dependencies}

%\subsubsection{GUI Dependencies}



\section{Building the Software}
\label{dev:build}

This section describes how to use the build system to build the software in different environments, 
but not how to modify the software or set it up to do something useful---this is covered in later 
sections.  This is because the build step varies between platforms, whereas modifying the software 
does not.

It is assumed that a ``zip file'' containing the source code has already been obtained and unzipped 
somewhere.

\subsection{In a Windows Environment}

\subsubsection{Building the F Model}

\begin{enumerate}

\item Open the Makefile from the source directory in a text editor and check that the 
\verb|WIN32_LIB_...| and \verb|WIN32_BIN_...| paths ``look right'' for how MinGW and MSYS was 
installed.

\item Open a Windows terminal prompt, for example by clicking the ``Start'' button and then 
``Run...'' and typing \verb|cmd.exe|, and change directory into the directory that was extracted 
from the zip file.  For example, if the zip file was \verb|DO3SE-src-F-20090620| and was extracted 
to ``My Documents'', your terminal session might look like this:
\begin{lstlisting}
C:\Documents and Settings\Alan>cd "My Documents\DO3SE-src-F-20090620"
C:\Documents and Settings\Alan\My Documents\DO3SE-src-F-20090620>dir
...
20/06/2009  18:53   <DIR>       .
20/06/2009  18:53   <DIR>       ..
20/06/2009  18:53   <DIR>       F
20/06/2009  18:53               Makefile
...
\end{lstlisting}
For Windows Vista and above, replace ``Documents and Settings'' and ``My Documents'' with ``Users'' 
and ``Documents''.

\item Run \verb|make PLATFORM=win32| to build the F model for 32-bit Windows.  If MSYS was not added 
to the \verb|PATH|, supply the full path to the \verb|make| executable:
\begin{lstlisting}
C:\...\DO3SE-src-F-20090620>c:\msys\1.0\make PLATFORM=win32
\end{lstlisting}

\item If there are no errors, the program \verb|dose.exe| should have be created in the current 
directory.
\begin{lstlisting}
C:\Documents and Settings\Alan\My Documents\DO3SE-src-F-20090620>dir
...
20/06/2009  18:53   <DIR>       .
20/06/2009  18:53   <DIR>       ..
20/06/2009  18:53   <DIR>       F
20/06/2009  18:53               Makefile
20/06/2009  18:56               dose.exe
...
\end{lstlisting}

\end{enumerate}


%\subsection{In a Linux Environment}



\section{Using the F Model---Quick Start}

The F model, when built, is a stand-alone command-line program.  To run it, make sure there is an 
input file of the correct filename in the current directory and then execute the program from the 
command line:
\begin{lstlisting}
C:\...\DO3SE-src-F-20090620>dir
...
20/06/2009  18:53   <DIR>       .
20/06/2009  18:53   <DIR>       ..
20/06/2009  18:53   <DIR>       F
20/06/2009  18:53               Makefile
20/06/2009  18:56               dose.exe
20/06/2009  18:56               input_newstyle.csv
...
C:\...\DO3SE-src-F-20090620>dose.exe
C:\...\DO3SE-src-F-20090620>dir
...
20/06/2009  18:53   <DIR>       .
20/06/2009  18:53   <DIR>       ..
20/06/2009  18:53   <DIR>       F
20/06/2009  18:53               Makefile
20/06/2009  18:56               dose.exe
20/06/2009  18:56               input_newstyle.csv
20/06/2009  18:59               output.csv
...
\end{lstlisting}

Alternatively, double-click on \verb|dose.exe| to run it---be aware that this will hide any errors 
that might have been visible on the command line.

If the output file (\verb|output.csv| by default) is empty, it is most likely that no output columns 
have been specified (see the next section).

\subsection{Configuring Input and Output}

The input and output of the F model is configured in \verb|F/dose.f90|.

\begin{itemize}

\item The input and output filenames are set near the bottom of the file, in the \verb|Run_DOSE| 
section.

\item The columns to include in the output are set in the \verb|WriteData| subroutine.  Uncomment a 
line to include that variable in the output.  The variables and their meanings are documented in 
\verb|F/variables.f90|.

\item The input format is defined in the \verb|ReadData| subroutine.  \emph{This should be used for 
guidance, but not modified.}  The variables and their meanings are documented in 
\verb|F/inputs.f90|.

\item Where certain calculations have multiple implementations, these can be chosen between in the 
\verb|Calculate| subroutine.

\end{itemize}

To change any of these options, edit \verb|F/dose.f90| as necessary and save it.  To use the 
changes, rebuild the model (see \ref{dev:build}) and run it again.  \emph{If the output filename has 
not been changed and the previous results have not been moved, they will be overwritten.}


%\subsection{Building the User Interface}

%foo bar

