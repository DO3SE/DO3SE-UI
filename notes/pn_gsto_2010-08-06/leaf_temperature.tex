\documentclass[a4paper]{article}

\usepackage{fullpage}
\usepackage{amsmath}
\usepackage{tabularx}
\usepackage[colorlinks=true,citecolor=black,filecolor=black,linkcolor=black,urlcolor=black]{hyperref}
\usepackage{natbib}
\usepackage{siunitx}

\title{Estimating leaf temperature}
\date{}

\begin{document}

\maketitle

% p_to_d(e_s(T) + delta(T), 18.01528, T+1) - p_to_d(e_s(T), 18.01528, T)

%\section{The Equation}

\section{``Simple'' method --- based on \citet{nobel83}}

\subsection{Derivation}

This method rearranges the equation for heat conduction from \citet[p.~361]{nobel83} to instead give 
the leaf temperature difference ($T_l - T_a$).  We assume that the rate of heat conduction ($J_C^H$) 
    is equivalent to net radiation ($R_N$).

\begin{eqnarray}
J_C^H & = & 2 K^\text{air} \frac{(T_l - T_a)}{\delta^\text{bl}} \\
T_l - T_a & = & \frac{\delta^\text{bl}J_C^H}{2 K^\text{air}}
\end{eqnarray}

The boundary layer thickness, $\delta^\text{bl}$, can be calculated (in metres) as

\begin{equation}
\delta^\text{bl} = \frac{c}{1000} \sqrt{\frac{l}{u}}
\end{equation}

where $c=4.0$ for flat leaves and $c=5.8$ for needles.

\subsection{Symbols}
\begin{tabularx}{\textwidth}{l | l | X}
Term & Units & Description \\
\hline
$J_C^H$ & \si{\watt\per\metre\squared} & Rate of heat conduction \\
$K^\text{air}$ & \si{\watt\per\metre\per\degreeCelsius} & Thermal conductivity of air, 0.0264 at 
30$^\circ$C \\
$T_l$ & \si{\degreeCelsius} & Leaf temperature \\
$T_a$ & \si{\degreeCelsius} & Air temperature \\
$\delta^\text{bl}$ & \si{\metre} & Boundary layer thickness \\
$l$ & \si{\metre} & Mean leaf length in downwind direction \\
$u$ & \si{\metre\per\second} & Wind speed \\
\end{tabularx}


\section{``Complex'' method --- based on \citet{thornley90}}


\subsection{Derivation}

\citet[p.~418]{thornley90} gives an equation for canopy/leaf temperature ($T_l$):
\begin{equation}
\label{e:main}
T_l =  T_a + \frac{\phi_N - \lambda g_w \Delta \rho_{va}}{\lambda \left(s g_w + \gamma g_a\right)}
\end{equation}

The latent heat of vapourisation of water ($\lambda$) can be estimated from the air temperature 
($T_a$) with the following (from \url{http://en.wikipedia.org/wiki/Latent_heat}):
\begin{equation}
\lambda = (-0.0000614342 T_a^3 + 0.00158927 T_a^2 - 2.36418 T_a + 2500.79) \times 1000
\end{equation}

Equation \eqref{e:main} depends on the vapour \emph{density} deficit ($\Delta \rho_{va}$), however 
usually vapour \emph{pressure} deficit is the measured quantity.  \citet[p.~409]{thornley90} gives 
an equation to convert vapour pressure to vapour density:
\begin{equation}
\label{e:ptod}
\rho_v = \frac{\rho \varepsilon p_v}{P}
\end{equation}

We can now find the vapour density deficit by calculating the saturation ($e_s$) and actual ($e$) 
    vapour pressures.  From \citet[p.~10]{monteith90}:
\begin{eqnarray}
\label{e:es}
e_s & = & 0.611 \times \text{exp}\left(\frac{17.27 T_a}{T_a + 237.3}\right) \\
e & = & e_s - \Delta p_v
\end{eqnarray}

Combining \eqref{e:ptod}, \eqref{e:es} and \eqref{e:e}:
\begin{equation}
\label{e:e}
\Delta \rho_{va} = \frac{\rho \varepsilon e_s}{P} - \frac{\rho \varepsilon e}{P} = \frac{\rho 
    \varepsilon (e_s - e)}{P} = \frac{\rho \varepsilon \Delta p_v}{P}
\end{equation}

The slope of the saturation vapour \emph{pressure} curve ($\Delta$) is obtained by differentiating 
the equation for $e_s$ (from \url{http://www.fao.org/docrep/X0490E/x0490e0k.htm}):
\begin{equation}
\label{e:deltavp}
\Delta = \frac{4098 e_s}{(T_a + 237.3)^2}
\end{equation}

Assuming this equation is correct\footnote{I have not checked this myself, there are many variations 
on the calculation for $e_s$ with different constants.}, the value can simply be converted to 
density in the same manner as for the vapour pressure itself\footnote{This method gives values 
    higher than those quoted by \citet[p.~408]{thornley90}, apparently by some scaling factor.}:
\begin{equation}
s = \frac{4098 \rho \varepsilon e_s}{P(T_a + 237.3)^2}
\end{equation}

The above equations depend on $\rho$, the density of dry air.  From \citet[p.~12]{monteith90}:
\begin{equation}
\rho = \frac{P M_A}{R (T_a + 273.15)}
\end{equation}

The psychrometric parameter ($\gamma$) is calculated as \citet[p.~407]{thornley90}:
\begin{equation}
\gamma = \frac{\rho c_p}{\lambda}
\end{equation}

The conductance for water vapour ($g_w$) is the result of the air ($g_a$) and canopy ($g_c$) 
    conductances in series \citet[p.~407]{thornley90}:
\begin{equation}
\frac{1}{g_w} = \frac{1}{g_a} + \frac{1}{g_c}
\end{equation}

We calculate $R_a$ in the DO3SE model, and the equation for $g_a$ given by 
\citet[p.~415]{thornley90} corresponds to the DO3SE calculation for $R_a$, therefore we can simply 
use:
\begin{equation}
g_a = \frac{1}{R_a}
\end{equation}

The $g_c$ term refers to the canopy stomatal conductance.  This is a core component of the DO3SE 
model, and we already have this in the form $G_{\text{sto,c}}$.

The only remaining component is $\phi_N$, the available energy balance.  This can be estimated as 
the net radiation ($R_N$) already calculated in the DO3SE model.\footnote{This assumption may be 
    incorrect.}


\subsection{Symbols}
\begin{tabularx}{\textwidth}{l | l | X}
Term & Units & Description \\
\hline
$T_l$ & \si{\degreeCelsius} & Leaf temperature \\
$T_a$ & \si{\degreeCelsius} & Air temperature \\
$\phi_N$ & \si{\watt\per\metre\squared} & Energy balance \\
$\lambda$ & \si{\joule\per\kilo\gram} & Latent heat vapourisation of water \\
$g_w$ & \si{\metre\per\second} & Conductance for water vapour \\
$g_a$ & \si{\metre\per\second} & Boundary layer conductance of water vapour \\
$g_c$ & \si{\metre\per\second} & Canopy conductance of water vapour \\
$\Delta \rho_{va}$ & \si{\kilo\gram\per\metre\cubed} & Atmospheric vapour density deficit \\
$\rho$ & \si{\kilo\gram\per\metre\cubed} & Density of dry air \\
$\varepsilon$ & --- & Ratio of molecular mass of water and air, 0.622 \\
$e_s$ & \si{\kilo\pascal} & Saturation vapour pressure at $T_a$\\
$e$ & \si{\kilo\pascal} & Actual vapour pressure \\
$\Delta p_v$ & \si{\kilo\pascal} & Vapour pressure deficit, $e_s - e$ \\
$P$ & \si{\kilo\pascal} & Atmospheric pressure \\
$s$ & \si{\kilo\gram\per\metre\cubed\per\kelvin} & Slope of saturation vapour density curve $\delta 
\rho'_v / \delta T (T = T_a)$ \\
$M_A$ & \si{\gram\per\mole} & Molecular weight of air, 29 \\
$R$ & \si{\joule\per\mole\per\kelvin} & Universal gas constant, 8.3144621 \\
$\gamma$ & \si{\kilo\gram\per\metre\cubed\per\kelvin} & Psychrometric parameter \\
$c_p$ & \si{\joule\per\kilo\gram\per\kelvin} & Specific heat capacity of dry air, 1010 \\
$R_a$ & \si{\second\per\metre} & Atmospheric resistance \\
\end{tabularx}

\begin{thebibliography}{}
\bibitem[Nobel(1983)]{nobel83}
    Nobel, P.S.,
    \emph{Biophysical Plant Physiology and Ecology}.
    WH Freeman and Company, New York, NY, 1983.
\bibitem[Thornley(1990)]{thornley90}
    Thornley, J.H.M. and Johnson, I.R.,
    \emph{Plant and Crop Modelling}.
    The Blackburn Press, Caldwell, NJ, 1990.
\bibitem[Monteith(1990)]{monteith90}
    Monteith, J.L. and Unsworth, M.,
    \emph{Principles of Environmental Physics}.
    St. Martin's Press, New York, NY, 2nd Edition, 1990.
\end{thebibliography}

\end{document}
